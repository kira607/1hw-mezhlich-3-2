\section{Демонстративное поведения}

\textbf{Произведение}: Ш. Перро, “Золушка”

\begin{quote}
    – А что, Золушка, хотелось бы тебе поехать на королевский бал? – спрашивали сестры, пока она причесывала их перед зеркалом.
    
    – Ах, что вы, сестрицы! Вы смеетесь надо мной! Разве меня пустят во дворец в этом платье и в этих башмаках!
    
    – Что правда, то правда. Вот была бы умора, если бы такая замарашка явилась на бал!
\end{quote}

Дочери мачехи Золушки демонстративно ее унизили, это проявление их защиты перед Золушкиной красотой и ее золотыми руками (чего у сестер не было, они завидовали).
