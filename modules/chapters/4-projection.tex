\section{Проекция}

\textbf{Произведение}: А. Грин, “Бродяга и начальник тюрьмы”

\begin{quote}
    “— Как ты лжешь! — сказал Пинкертон. — Зачем ты лжешь?
Прежде чем ответить, бродяга сделал несколько ударов киркой, затем оперся на нее с видом отдыхающего скульптора.
— Это не ложь, — грустно сказал он. — Боже мой! Какая весна! ”
\end{quote}

Пинкертон обвиняет собеседника во лжи, при этом не зная сути, а просто делая выводы, то есть приписывает свои же убеждения или мысли и взгляды на мир. 
