\section{Рационализация}

\textbf{Произведение}: Л. Толстой, “Война и Мир”

\begin{quote}
    --- Моя жена, – продолжал князь Андрей, – прекрасная женщина. 
    Это одна из тех редких женщин, 
    с которою можно быть покойным за свою честь; 
    но, Боже мой, чего бы я не дал теперь, чтобы не быть женатым! 
    Это я тебе одному и первому говорю, 
    потому что я люблю тебя.
    
    Гостиные, сплетни, балы, тщеславие, ничтожество — вот заколдованный круг, из которого я не могу выйти. Я теперь отправляюсь на войну, на величайшую войну, какая только бывала, а я ничего не знаю и никуда не гожусь. <…> Эгоизм, тщеславие, тупоумие, ничтожество во всем — вот женщины, когда они показываются так, как они есть. Посмотришь на них в свете, кажется, что что-то есть, а ничего, ничего, ничего! Да, не женись, душа моя, не женись.”
\end{quote}

Герой рассуждает, приходя к выводу, что его ожидания от женитьбы не оправдались и он пытается убедить друга, чтобы тот не женился. В этом проявляется механизм защиты князя Андрея.
